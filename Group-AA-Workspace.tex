% Options for packages loaded elsewhere
\PassOptionsToPackage{unicode}{hyperref}
\PassOptionsToPackage{hyphens}{url}
%
\documentclass[
]{article}
\usepackage{amsmath,amssymb}
\usepackage{lmodern}
\usepackage{iftex}
\ifPDFTeX
  \usepackage[T1]{fontenc}
  \usepackage[utf8]{inputenc}
  \usepackage{textcomp} % provide euro and other symbols
\else % if luatex or xetex
  \usepackage{unicode-math}
  \defaultfontfeatures{Scale=MatchLowercase}
  \defaultfontfeatures[\rmfamily]{Ligatures=TeX,Scale=1}
\fi
% Use upquote if available, for straight quotes in verbatim environments
\IfFileExists{upquote.sty}{\usepackage{upquote}}{}
\IfFileExists{microtype.sty}{% use microtype if available
  \usepackage[]{microtype}
  \UseMicrotypeSet[protrusion]{basicmath} % disable protrusion for tt fonts
}{}
\makeatletter
\@ifundefined{KOMAClassName}{% if non-KOMA class
  \IfFileExists{parskip.sty}{%
    \usepackage{parskip}
  }{% else
    \setlength{\parindent}{0pt}
    \setlength{\parskip}{6pt plus 2pt minus 1pt}}
}{% if KOMA class
  \KOMAoptions{parskip=half}}
\makeatother
\usepackage{xcolor}
\IfFileExists{xurl.sty}{\usepackage{xurl}}{} % add URL line breaks if available
\IfFileExists{bookmark.sty}{\usepackage{bookmark}}{\usepackage{hyperref}}
\hypersetup{
  pdftitle={Group AA Milestone 3},
  hidelinks,
  pdfcreator={LaTeX via pandoc}}
\urlstyle{same} % disable monospaced font for URLs
\usepackage[margin=1in]{geometry}
\usepackage{color}
\usepackage{fancyvrb}
\newcommand{\VerbBar}{|}
\newcommand{\VERB}{\Verb[commandchars=\\\{\}]}
\DefineVerbatimEnvironment{Highlighting}{Verbatim}{commandchars=\\\{\}}
% Add ',fontsize=\small' for more characters per line
\usepackage{framed}
\definecolor{shadecolor}{RGB}{248,248,248}
\newenvironment{Shaded}{\begin{snugshade}}{\end{snugshade}}
\newcommand{\AlertTok}[1]{\textcolor[rgb]{0.94,0.16,0.16}{#1}}
\newcommand{\AnnotationTok}[1]{\textcolor[rgb]{0.56,0.35,0.01}{\textbf{\textit{#1}}}}
\newcommand{\AttributeTok}[1]{\textcolor[rgb]{0.77,0.63,0.00}{#1}}
\newcommand{\BaseNTok}[1]{\textcolor[rgb]{0.00,0.00,0.81}{#1}}
\newcommand{\BuiltInTok}[1]{#1}
\newcommand{\CharTok}[1]{\textcolor[rgb]{0.31,0.60,0.02}{#1}}
\newcommand{\CommentTok}[1]{\textcolor[rgb]{0.56,0.35,0.01}{\textit{#1}}}
\newcommand{\CommentVarTok}[1]{\textcolor[rgb]{0.56,0.35,0.01}{\textbf{\textit{#1}}}}
\newcommand{\ConstantTok}[1]{\textcolor[rgb]{0.00,0.00,0.00}{#1}}
\newcommand{\ControlFlowTok}[1]{\textcolor[rgb]{0.13,0.29,0.53}{\textbf{#1}}}
\newcommand{\DataTypeTok}[1]{\textcolor[rgb]{0.13,0.29,0.53}{#1}}
\newcommand{\DecValTok}[1]{\textcolor[rgb]{0.00,0.00,0.81}{#1}}
\newcommand{\DocumentationTok}[1]{\textcolor[rgb]{0.56,0.35,0.01}{\textbf{\textit{#1}}}}
\newcommand{\ErrorTok}[1]{\textcolor[rgb]{0.64,0.00,0.00}{\textbf{#1}}}
\newcommand{\ExtensionTok}[1]{#1}
\newcommand{\FloatTok}[1]{\textcolor[rgb]{0.00,0.00,0.81}{#1}}
\newcommand{\FunctionTok}[1]{\textcolor[rgb]{0.00,0.00,0.00}{#1}}
\newcommand{\ImportTok}[1]{#1}
\newcommand{\InformationTok}[1]{\textcolor[rgb]{0.56,0.35,0.01}{\textbf{\textit{#1}}}}
\newcommand{\KeywordTok}[1]{\textcolor[rgb]{0.13,0.29,0.53}{\textbf{#1}}}
\newcommand{\NormalTok}[1]{#1}
\newcommand{\OperatorTok}[1]{\textcolor[rgb]{0.81,0.36,0.00}{\textbf{#1}}}
\newcommand{\OtherTok}[1]{\textcolor[rgb]{0.56,0.35,0.01}{#1}}
\newcommand{\PreprocessorTok}[1]{\textcolor[rgb]{0.56,0.35,0.01}{\textit{#1}}}
\newcommand{\RegionMarkerTok}[1]{#1}
\newcommand{\SpecialCharTok}[1]{\textcolor[rgb]{0.00,0.00,0.00}{#1}}
\newcommand{\SpecialStringTok}[1]{\textcolor[rgb]{0.31,0.60,0.02}{#1}}
\newcommand{\StringTok}[1]{\textcolor[rgb]{0.31,0.60,0.02}{#1}}
\newcommand{\VariableTok}[1]{\textcolor[rgb]{0.00,0.00,0.00}{#1}}
\newcommand{\VerbatimStringTok}[1]{\textcolor[rgb]{0.31,0.60,0.02}{#1}}
\newcommand{\WarningTok}[1]{\textcolor[rgb]{0.56,0.35,0.01}{\textbf{\textit{#1}}}}
\usepackage{graphicx}
\makeatletter
\def\maxwidth{\ifdim\Gin@nat@width>\linewidth\linewidth\else\Gin@nat@width\fi}
\def\maxheight{\ifdim\Gin@nat@height>\textheight\textheight\else\Gin@nat@height\fi}
\makeatother
% Scale images if necessary, so that they will not overflow the page
% margins by default, and it is still possible to overwrite the defaults
% using explicit options in \includegraphics[width, height, ...]{}
\setkeys{Gin}{width=\maxwidth,height=\maxheight,keepaspectratio}
% Set default figure placement to htbp
\makeatletter
\def\fps@figure{htbp}
\makeatother
\setlength{\emergencystretch}{3em} % prevent overfull lines
\providecommand{\tightlist}{%
  \setlength{\itemsep}{0pt}\setlength{\parskip}{0pt}}
\setcounter{secnumdepth}{-\maxdimen} % remove section numbering
\ifLuaTeX
  \usepackage{selnolig}  % disable illegal ligatures
\fi

\title{Group AA Milestone 3}
\author{}
\date{\vspace{-2.5em}2022-11-01}

\begin{document}
\maketitle

\hypertarget{r-markdown}{%
\subsection{R Markdown}\label{r-markdown}}

This is an R Markdown document. Markdown is a simple formatting syntax
for authoring HTML, PDF, and MS Word documents. For more details on
using R Markdown see \url{http://rmarkdown.rstudio.com}.

When you click the \textbf{Knit} button a document will be generated
that includes both content as well as the output of any embedded R code
chunks within the document. You can embed an R code chunk like this:

\begin{Shaded}
\begin{Highlighting}[]
\NormalTok{hi }\OtherTok{\textless{}{-}} \StringTok{"hi group!"}
\FunctionTok{print}\NormalTok{(hi)}
\end{Highlighting}
\end{Shaded}

\begin{verbatim}
## [1] "hi group!"
\end{verbatim}

\newpage

\hypertarget{part-1-describing-the-dataset}{%
\subsection{\texorpdfstring{Part 1: Describing the dataset
\n}{Part 1: Describing the dataset }}\label{part-1-describing-the-dataset}}

\begin{itemize}
\tightlist
\item
  What is the data source? (1-2 sentences on where the data is coming
  from, dates included, etc.)
\end{itemize}

The data comes from the 2011 California Smokers' Cohort (CSC) and was
the ninth of a series of triennial surveys called the California Tobacco
Surveys (CTS) conducted since 1990. It was sponsored by the State of
California's Department of Public Health through a contract with the
University of California at San Diego (UCSD). Data collection for CLSS
began on July 8, 2011 and was completed on December 8, 2011.

\begin{itemize}
\tightlist
\item
  How does the dataset relate to the group problem statement and
  question? \n
\end{itemize}

Smoking has been shown to lead to various poor health outcomes. However,
some smokers are more prone than others to these adverse effects. The
question is what characteristics and behaviors among smokers in
California led to adverse health outcomes.

\newpage

\hypertarget{part-2-import-statement}{%
\subsection{\texorpdfstring{Part 2: Import statement
\n}{Part 2: Import statement }}\label{part-2-import-statement}}

\begin{itemize}
\tightlist
\item
  Use appropriate import function and package based on the type of file
  \n
\item
  Utilize function arguments to control relevant components (i.e.~change
  column types, column names, missing values, etc.) \n
\item
  Document the import process \n
\end{itemize}

\begin{Shaded}
\begin{Highlighting}[]
\CommentTok{\#import CA Smoker data set}
\NormalTok{ca\_smoker\_info }\OtherTok{\textless{}{-}} \FunctionTok{read\_csv}\NormalTok{(}\StringTok{"\textasciitilde{}/PHW251\_Fall2022/phw251\_projectdata/ca\_csc\_smoker\_data.csv"}\NormalTok{)}
\end{Highlighting}
\end{Shaded}

\begin{verbatim}
## Rows: 1000 Columns: 156
## -- Column specification --------------------------------------------------------
## Delimiter: ","
## chr (152): RIGHTSEX, smokstat, ACIG100, DOSMOKE, HOWMANY, SMOK6NUM, SMOK6UNI...
## dbl   (3): psraid, nosmknum1, quitoffn
## lgl   (1): QUITINTNFORM
## 
## i Use `spec()` to retrieve the full column specification for this data.
## i Specify the column types or set `show_col_types = FALSE` to quiet this message.
\end{verbatim}

\begin{Shaded}
\begin{Highlighting}[]
\CommentTok{\#tidying data}
\NormalTok{ca\_smoker\_selected }\OtherTok{\textless{}{-}}\NormalTok{ ca\_smoker\_info }\SpecialCharTok{\%\textgreater{}\%} \FunctionTok{select}\NormalTok{(}\FunctionTok{c}\NormalTok{(psraid,smokstat,HOWMANY, SMOK6NUM, SMOK6UNI)) }\SpecialCharTok{\%\textgreater{}\%} 
\FunctionTok{rename}\NormalTok{(}\AttributeTok{ID =}\NormalTok{ psraid, }\AttributeTok{smoking\_status =}\NormalTok{ smokstat, }\AttributeTok{howmany =}\NormalTok{ HOWMANY, }\AttributeTok{smok6num =}\NormalTok{ SMOK6NUM, }\AttributeTok{smok6uni =}\NormalTok{ SMOK6UNI) }\SpecialCharTok{\%\textgreater{}\%}
\FunctionTok{mutate}\NormalTok{(}\AttributeTok{pack\_year =}\NormalTok{ howmany)}

\CommentTok{\#Hi this is Danni, for the calculation of pack{-}year, Im not quite sure which columns\textquotesingle{} data and which equation should I use. I just put the new column there and leave it with the value of "howmany", we can discuss more about it and I\textquotesingle{}ll redo the pack{-}year column}
\end{Highlighting}
\end{Shaded}

\newpage

\begin{Shaded}
\begin{Highlighting}[]
\CommentTok{\#import CA smoker disease outcome and race data set}
\NormalTok{ca\_outcome\_race }\OtherTok{\textless{}{-}} \FunctionTok{read\_csv}\NormalTok{(}\StringTok{"\textasciitilde{}/PHW251\_Fall2022/phw251\_projectdata/ca\_csc\_outcome\_race\_data.csv"}\NormalTok{)}
\end{Highlighting}
\end{Shaded}

\begin{verbatim}
## Rows: 1000 Columns: 89
## -- Column specification --------------------------------------------------------
## Delimiter: ","
## chr (81): ID, INCARS, BANAGREE, CASINSMK, CASMOKES, HHSMOKNU, ACQSMOKE, LIVE...
## dbl  (6): ACTIVHRS, ACTIVMIN, HTINFEET, HTINCHES, WGTINLBS, AGEUS
## lgl  (2): HTCENTIM, WGTINKILOS
## 
## i Use `spec()` to retrieve the full column specification for this data.
## i Specify the column types or set `show_col_types = FALSE` to quiet this message.
\end{verbatim}

\begin{Shaded}
\begin{Highlighting}[]
\CommentTok{\#tidying data}
\NormalTok{ca\_outcome\_race\_selected }\OtherTok{\textless{}{-}}\NormalTok{ ca\_outcome\_race }\SpecialCharTok{\%\textgreater{}\%} \FunctionTok{select}\NormalTok{(}\FunctionTok{c}\NormalTok{(ID, SOCIAL, ASTHMA, }
\NormalTok{                                                         HEARTDIS, DIABETES, }
\NormalTok{                                                         OTHMENILL, INCOME, }
\NormalTok{                                                         race01, race02, race03, }
\NormalTok{                                                         race04, race05, race06, }
\NormalTok{                                                         race07, race08, race09, }
\NormalTok{                                                         race10, race11, race12, }
\NormalTok{                                                         race13,race14, race15)) }\SpecialCharTok{\%\textgreater{}\%}
\FunctionTok{rename}\NormalTok{ (}\AttributeTok{social =}\NormalTok{ SOCIAL, }\AttributeTok{asthma =}\NormalTok{ ASTHMA, }\AttributeTok{heartdis =}\NormalTok{ HEARTDIS, }\AttributeTok{diabetes =}\NormalTok{ DIABETES, }
        \AttributeTok{othmenill =}\NormalTok{ OTHMENILL, }\AttributeTok{income =}\NormalTok{ INCOME)}
\end{Highlighting}
\end{Shaded}

\newpage

\begin{Shaded}
\begin{Highlighting}[]
\CommentTok{\#remove "DIS" \& "STAT" in the "ID" column }
\NormalTok{ca\_outcome\_race\_selected}\SpecialCharTok{$}\NormalTok{ID }\OtherTok{\textless{}{-}} \FunctionTok{gsub}\NormalTok{(}\StringTok{"DIS"}\NormalTok{,}\StringTok{""}\NormalTok{,}\FunctionTok{as.character}\NormalTok{(ca\_outcome\_race\_selected}\SpecialCharTok{$}\NormalTok{ID))}
\NormalTok{ca\_outcome\_race\_selected}\SpecialCharTok{$}\NormalTok{ID }\OtherTok{\textless{}{-}} \FunctionTok{gsub}\NormalTok{(}\StringTok{"STAT"}\NormalTok{,}\StringTok{""}\NormalTok{,}\FunctionTok{as.character}\NormalTok{(ca\_outcome\_race\_selected}\SpecialCharTok{$}\NormalTok{ID))}

\CommentTok{\#joining two data sets by participant\textquotesingle{}s unique ID}
\NormalTok{ca\_smoker\_outcome }\OtherTok{\textless{}{-}} \FunctionTok{merge}\NormalTok{(}\AttributeTok{x =}\NormalTok{ ca\_smoker\_selected, }\AttributeTok{y =}\NormalTok{ ca\_outcome\_race\_selected, }\AttributeTok{by =} \StringTok{"ID"}\NormalTok{)}


\CommentTok{\#Hi this is Danni, here is all the columns we need in one table, (pls let me know if any column needed is not here :) Also, the race part seems unclear, maybe we can summarize it in one column that including 15 races. For now, we can leave it here and I\textquotesingle{}ll try how to combine all 15 race columns in one column later)}

\CommentTok{\#Hi this is Maddy! I combined the races into one column, please see code below :) }
\end{Highlighting}
\end{Shaded}

\newpage

\begin{Shaded}
\begin{Highlighting}[]
\CommentTok{\#use mutate to combine 15 binary race columns into one categorical variable called "race"}

\NormalTok{ca\_smoker\_outcome }\OtherTok{\textless{}{-}}\NormalTok{ ca\_smoker\_outcome }\SpecialCharTok{\%\textgreater{}\%} 
  \FunctionTok{mutate}\NormalTok{(}\AttributeTok{race =} \FunctionTok{case\_when}\NormalTok{(race01 }\SpecialCharTok{==} \StringTok{"Yes"} \SpecialCharTok{\textasciitilde{}} \StringTok{"White"}\NormalTok{,}
\NormalTok{                          race02 }\SpecialCharTok{==} \StringTok{"Yes"} \SpecialCharTok{\textasciitilde{}} \StringTok{"Black"}\NormalTok{,}
\NormalTok{                          race03 }\SpecialCharTok{==} \StringTok{"Yes"} \SpecialCharTok{\textasciitilde{}} \StringTok{"Japanese"}\NormalTok{,}
\NormalTok{                          race04 }\SpecialCharTok{==} \StringTok{"Yes"} \SpecialCharTok{\textasciitilde{}} \StringTok{"Chinese"}\NormalTok{,}
\NormalTok{                          race05 }\SpecialCharTok{==} \StringTok{"Yes"} \SpecialCharTok{\textasciitilde{}} \StringTok{"Filipino"}\NormalTok{,}
\NormalTok{                          race06 }\SpecialCharTok{==} \StringTok{"Yes"} \SpecialCharTok{\textasciitilde{}} \StringTok{"Korean"}\NormalTok{,}
\NormalTok{                          race12 }\SpecialCharTok{==} \StringTok{"Yes"} \SpecialCharTok{\textasciitilde{}} \StringTok{"Vietnamese"}\NormalTok{,}
\NormalTok{                          race07 }\SpecialCharTok{==} \StringTok{"Yes"} \SpecialCharTok{\textasciitilde{}} \StringTok{"Other Asian Pacific Islander"}\NormalTok{,}
\NormalTok{                          race08 }\SpecialCharTok{==} \StringTok{"Yes"} \SpecialCharTok{\textasciitilde{}} \StringTok{"American Indian Alaska Native"}\NormalTok{,}
\NormalTok{                          race09 }\SpecialCharTok{==} \StringTok{"Yes"} \SpecialCharTok{\textasciitilde{}} \StringTok{"Mexican"}\NormalTok{,}
\NormalTok{                          race10 }\SpecialCharTok{==} \StringTok{"Yes"} \SpecialCharTok{\textasciitilde{}} \StringTok{"Hispanic or Latino"}\NormalTok{,}
\NormalTok{                          race11 }\SpecialCharTok{==} \StringTok{"Yes"} \SpecialCharTok{\textasciitilde{}} \StringTok{"Other"}\NormalTok{,}
\NormalTok{                          race13 }\SpecialCharTok{==} \StringTok{"Yes"} \SpecialCharTok{\textasciitilde{}} \StringTok{"Asian Indian"}\NormalTok{,}
\NormalTok{                          race14 }\SpecialCharTok{==} \StringTok{"Yes"} \SpecialCharTok{\textasciitilde{}} \StringTok{"Refused"}\NormalTok{,}
\NormalTok{                          race15 }\SpecialCharTok{==} \StringTok{"Yes"} \SpecialCharTok{\textasciitilde{}} \StringTok{"Don\textquotesingle{}t Know"}\NormalTok{))}

\CommentTok{\#drop leftover binary race columns }

\NormalTok{ca\_smoker\_outcome }\OtherTok{\textless{}{-}} \FunctionTok{select}\NormalTok{(ca\_smoker\_outcome,}\SpecialCharTok{{-}}\NormalTok{race01,}
                                 \SpecialCharTok{{-}}\NormalTok{race02,}
                                 \SpecialCharTok{{-}}\NormalTok{race03,}
                                 \SpecialCharTok{{-}}\NormalTok{race04,}
                                 \SpecialCharTok{{-}}\NormalTok{race05,}
                                 \SpecialCharTok{{-}}\NormalTok{race06,}
                                 \SpecialCharTok{{-}}\NormalTok{race07,}
                                 \SpecialCharTok{{-}}\NormalTok{race08,}
                                 \SpecialCharTok{{-}}\NormalTok{race09,}
                                 \SpecialCharTok{{-}}\NormalTok{race10,}
                                 \SpecialCharTok{{-}}\NormalTok{race11,}
                                 \SpecialCharTok{{-}}\NormalTok{race12,}
                                 \SpecialCharTok{{-}}\NormalTok{race13,}
                                 \SpecialCharTok{{-}}\NormalTok{race14,}
                                 \SpecialCharTok{{-}}\NormalTok{race15)}
\end{Highlighting}
\end{Shaded}

\newpage

\hypertarget{part-3-identify-data-types-for-5-data-elementscolumnsvariables}{%
\subsection{\texorpdfstring{Part 3: Identify data types for 5+ data
elements/columns/variables
\n}{Part 3: Identify data types for 5+ data elements/columns/variables }}\label{part-3-identify-data-types-for-5-data-elementscolumnsvariables}}

\begin{itemize}
\tightlist
\item
  Identify 5+ data elements required for your specified scenario. If
  \textless5 elements are required to complete the analysis, please
  choose additional variables of interest in the data set to explore in
  this milestone. \n
\item
  Utilize functions or resources in RStudio to determine the types of
  each data element (i.e.~character, numeric, factor) \n
\item
  Identify the desired type/format for each variable---will you need to
  convert any columns to numeric or another type? \n
\end{itemize}

Five variables of interest: 1. Smoking status 2. Race 3. Income 4. Heart
Disease 5. Pack years

\begin{Shaded}
\begin{Highlighting}[]
\CommentTok{\#identify types of each data element}
\FunctionTok{str}\NormalTok{(ca\_smoker\_outcome)}
\end{Highlighting}
\end{Shaded}

\begin{verbatim}
## 'data.frame':    1000 obs. of  13 variables:
##  $ ID            : num  1e+05 1e+05 1e+05 1e+05 1e+05 ...
##  $ smoking_status: chr  "Current daily smoker" "Current daily smoker" "Current nondaily smoker" "Current daily smoker" ...
##  $ howmany       : chr  "30" "20" "1" "15" ...
##  $ smok6num      : chr  "36" "25" NA "20" ...
##  $ smok6uni      : chr  "Years" "Years" NA "Years" ...
##  $ pack_year     : chr  "30" "20" "1" "15" ...
##  $ social        : chr  "No" "Yes" "Yes" "Yes" ...
##  $ asthma        : chr  "No" "No" "No" "Yes" ...
##  $ heartdis      : chr  "Yes" "No" "No" "No" ...
##  $ diabetes      : chr  "No" "No" "No" "No" ...
##  $ othmenill     : chr  "No" "No" "No" "No" ...
##  $ income        : chr  "$30,001 to $50,000" "$20,000 or less" "$30,001 to $50,000" "$20,001 to $30,000" ...
##  $ race          : chr  "White" "White" "White" "White" ...
\end{verbatim}

\begin{Shaded}
\begin{Highlighting}[]
\CommentTok{\#convert data types to appropriate type in new column such as as.factor; as.numeric; as.character}
\NormalTok{ca\_smoker\_outcome }\OtherTok{\textless{}{-}}\NormalTok{ ca\_smoker\_outcome }\SpecialCharTok{\%\textgreater{}\%} \FunctionTok{mutate}\NormalTok{(}\AttributeTok{new\_howmany =} \FunctionTok{as.numeric}\NormalTok{(howmany)) }\SpecialCharTok{\%\textgreater{}\%}
  \FunctionTok{mutate}\NormalTok{(}\AttributeTok{new\_smoking\_status=} \FunctionTok{as.factor}\NormalTok{(smoking\_status)) }\SpecialCharTok{\%\textgreater{}\%} 
  \FunctionTok{mutate}\NormalTok{(}\AttributeTok{new\_smok6num =} \FunctionTok{as.numeric}\NormalTok{(smok6num)) }\SpecialCharTok{\%\textgreater{}\%} 
  \FunctionTok{mutate}\NormalTok{(}\AttributeTok{new\_smok6uni =} \FunctionTok{as.factor}\NormalTok{(smok6uni)) }\SpecialCharTok{\%\textgreater{}\%}
  \FunctionTok{mutate}\NormalTok{(}\AttributeTok{new\_pack\_year =} \FunctionTok{as.numeric}\NormalTok{(pack\_year)) }\SpecialCharTok{\%\textgreater{}\%} 
  \FunctionTok{mutate}\NormalTok{(}\AttributeTok{new\_social =} \FunctionTok{as.factor}\NormalTok{(social)) }\SpecialCharTok{\%\textgreater{}\%} 
  \FunctionTok{mutate}\NormalTok{(}\AttributeTok{new\_asthma =} \FunctionTok{as.factor}\NormalTok{(asthma)) }\SpecialCharTok{\%\textgreater{}\%} 
  \FunctionTok{mutate}\NormalTok{(}\AttributeTok{new\_heartdis =} \FunctionTok{as.factor}\NormalTok{(heartdis)) }\SpecialCharTok{\%\textgreater{}\%} 
  \FunctionTok{mutate}\NormalTok{(}\AttributeTok{new\_diabetes =} \FunctionTok{as.factor}\NormalTok{(heartdis)) }\SpecialCharTok{\%\textgreater{}\%} 
  \FunctionTok{mutate}\NormalTok{(}\AttributeTok{new\_othmenill =} \FunctionTok{as.factor}\NormalTok{(othmenill)) }\SpecialCharTok{\%\textgreater{}\%} 
  \FunctionTok{mutate}\NormalTok{(}\AttributeTok{new\_income =} \FunctionTok{as.factor}\NormalTok{(income))}
\end{Highlighting}
\end{Shaded}

\begin{verbatim}
## Warning in mask$eval_all_mutate(quo): NAs introduced by coercion

## Warning in mask$eval_all_mutate(quo): NAs introduced by coercion

## Warning in mask$eval_all_mutate(quo): NAs introduced by coercion
\end{verbatim}

\begin{Shaded}
\begin{Highlighting}[]
\FunctionTok{summary}\NormalTok{(ca\_smoker\_outcome}\SpecialCharTok{$}\NormalTok{new\_howmany)}
\end{Highlighting}
\end{Shaded}

\begin{verbatim}
##    Min. 1st Qu.  Median    Mean 3rd Qu.    Max.    NA's 
##    1.00    7.00   12.00   13.89   20.00   60.00      10
\end{verbatim}

\newpage

\hypertarget{part-4-provide-a-basic-description-of-the-5-data-elements}{%
\subsection{\texorpdfstring{Part 4: Provide a basic description of the
5+ data elements
\n}{Part 4: Provide a basic description of the 5+ data elements }}\label{part-4-provide-a-basic-description-of-the-5-data-elements}}

\begin{itemize}
\tightlist
\item
  Numeric: mean, median, range \n
\item
  Character: unique values/categories \n
\item
  Or any other descriptives that will be useful to the analysis \n
\end{itemize}

\emph{Smoking status:}

\begin{Shaded}
\begin{Highlighting}[]
\CommentTok{\#number of unique categories }
\NormalTok{ca\_smoker\_outcome }\SpecialCharTok{\%\textgreater{}\%} \FunctionTok{summarize}\NormalTok{(}\FunctionTok{n\_distinct}\NormalTok{(smoking\_status))}
\end{Highlighting}
\end{Shaded}

\begin{verbatim}
##   n_distinct(smoking_status)
## 1                          2
\end{verbatim}

\begin{Shaded}
\begin{Highlighting}[]
\CommentTok{\#names of unique categories }
\NormalTok{ca\_smoker\_outcome }\SpecialCharTok{\%\textgreater{}\%} \FunctionTok{summarize}\NormalTok{(}\FunctionTok{unique}\NormalTok{(smoking\_status))}
\end{Highlighting}
\end{Shaded}

\begin{verbatim}
##    unique(smoking_status)
## 1    Current daily smoker
## 2 Current nondaily smoker
\end{verbatim}

\begin{Shaded}
\begin{Highlighting}[]
\CommentTok{\#tabulate smoking status}
\FunctionTok{table}\NormalTok{(ca\_smoker\_outcome}\SpecialCharTok{$}\NormalTok{smoking\_status)}
\end{Highlighting}
\end{Shaded}

\begin{verbatim}
## 
##    Current daily smoker Current nondaily smoker 
##                     837                     163
\end{verbatim}

\emph{Race:}

\begin{Shaded}
\begin{Highlighting}[]
\CommentTok{\#number of unique categories }
\NormalTok{ca\_smoker\_outcome }\SpecialCharTok{\%\textgreater{}\%} \FunctionTok{summarize}\NormalTok{(}\FunctionTok{n\_distinct}\NormalTok{(race))}
\end{Highlighting}
\end{Shaded}

\begin{verbatim}
##   n_distinct(race)
## 1               14
\end{verbatim}

\begin{Shaded}
\begin{Highlighting}[]
\CommentTok{\#names of unique categories }
\NormalTok{ca\_smoker\_outcome }\SpecialCharTok{\%\textgreater{}\%} \FunctionTok{summarize}\NormalTok{(}\FunctionTok{unique}\NormalTok{(race))}
\end{Highlighting}
\end{Shaded}

\begin{verbatim}
##                     unique(race)
## 1                          White
## 2                          Black
## 3   Other Asian Pacific Islander
## 4  American Indian Alaska Native
## 5             Hispanic or Latino
## 6                        Mexican
## 7                   Asian Indian
## 8                       Filipino
## 9                       Japanese
## 10                    Don't Know
## 11                       Chinese
## 12                       Refused
## 13                         Other
## 14                    Vietnamese
\end{verbatim}

\begin{Shaded}
\begin{Highlighting}[]
\CommentTok{\#tabulate smoking status}
\FunctionTok{table}\NormalTok{(ca\_smoker\_outcome}\SpecialCharTok{$}\NormalTok{race)}
\end{Highlighting}
\end{Shaded}

\begin{verbatim}
## 
## American Indian Alaska Native                  Asian Indian 
##                            40                             1 
##                         Black                       Chinese 
##                            78                             7 
##                    Don't Know                      Filipino 
##                             2                             8 
##            Hispanic or Latino                      Japanese 
##                            17                             6 
##                       Mexican                         Other 
##                            19                             3 
##  Other Asian Pacific Islander                       Refused 
##                             6                             7 
##                    Vietnamese                         White 
##                             2                           804
\end{verbatim}

\emph{Income:}

\begin{Shaded}
\begin{Highlighting}[]
\CommentTok{\#number of unique categories }
\NormalTok{ca\_smoker\_outcome }\SpecialCharTok{\%\textgreater{}\%} \FunctionTok{summarize}\NormalTok{(}\FunctionTok{n\_distinct}\NormalTok{(income))}
\end{Highlighting}
\end{Shaded}

\begin{verbatim}
##   n_distinct(income)
## 1                  9
\end{verbatim}

\begin{Shaded}
\begin{Highlighting}[]
\CommentTok{\#names of unique categories }
\NormalTok{ca\_smoker\_outcome }\SpecialCharTok{\%\textgreater{}\%} \FunctionTok{summarize}\NormalTok{(}\FunctionTok{unique}\NormalTok{(income))}
\end{Highlighting}
\end{Shaded}

\begin{verbatim}
##             unique(income)
## 1       $30,001 to $50,000
## 2          $20,000 or less
## 3       $20,001 to $30,000
## 4     $100,001 to $150,000
## 5       $50,001 to $75,000
## 6            Over $150,000
## 7      $75,001 to $100,000
## 8    (DO NOT READ) Refused
## 9 (DO NOT READ) Don't know
\end{verbatim}

\begin{Shaded}
\begin{Highlighting}[]
\CommentTok{\#tabulate income}
\FunctionTok{table}\NormalTok{(ca\_smoker\_outcome}\SpecialCharTok{$}\NormalTok{income)}
\end{Highlighting}
\end{Shaded}

\begin{verbatim}
## 
## (DO NOT READ) Don't know    (DO NOT READ) Refused     $100,001 to $150,000 
##                       14                       48                       83 
##          $20,000 or less       $20,001 to $30,000       $30,001 to $50,000 
##                      243                      139                      182 
##       $50,001 to $75,000      $75,001 to $100,000            Over $150,000 
##                      157                       90                       44
\end{verbatim}

\emph{Heart Disease:}

\begin{Shaded}
\begin{Highlighting}[]
\CommentTok{\#number of unique categories }
\NormalTok{ca\_smoker\_outcome }\SpecialCharTok{\%\textgreater{}\%} \FunctionTok{summarize}\NormalTok{(}\FunctionTok{n\_distinct}\NormalTok{(heartdis))}
\end{Highlighting}
\end{Shaded}

\begin{verbatim}
##   n_distinct(heartdis)
## 1                    3
\end{verbatim}

\begin{Shaded}
\begin{Highlighting}[]
\CommentTok{\#names of unique categories }
\NormalTok{ca\_smoker\_outcome }\SpecialCharTok{\%\textgreater{}\%} \FunctionTok{summarize}\NormalTok{(}\FunctionTok{unique}\NormalTok{(heartdis))}
\end{Highlighting}
\end{Shaded}

\begin{verbatim}
##           unique(heartdis)
## 1                      Yes
## 2                       No
## 3 (DO NOT READ) Don't know
\end{verbatim}

\begin{Shaded}
\begin{Highlighting}[]
\CommentTok{\#tabulate income}
\FunctionTok{table}\NormalTok{(ca\_smoker\_outcome}\SpecialCharTok{$}\NormalTok{heartdis)}
\end{Highlighting}
\end{Shaded}

\begin{verbatim}
## 
## (DO NOT READ) Don't know                       No                      Yes 
##                        3                      916                       81
\end{verbatim}

\emph{Pack Years:}

\begin{Shaded}
\begin{Highlighting}[]
\CommentTok{\#look at minimum, median, mean, maximum, and \# of NAs in pack year}
\FunctionTok{summary}\NormalTok{(ca\_smoker\_outcome}\SpecialCharTok{$}\NormalTok{new\_pack\_year)}
\end{Highlighting}
\end{Shaded}

\begin{verbatim}
##    Min. 1st Qu.  Median    Mean 3rd Qu.    Max.    NA's 
##    1.00    7.00   12.00   13.89   20.00   60.00      10
\end{verbatim}

\emph{Milestone 3} - Subset rows or columns, as needed - Create new
variables needed for analysis (minimum 2) - New variables should be
created based on existing columns; for example - Calculating a rate -
Combining character strings - Reordering income to CA - low, med, high -
Pack years -\textgreater{} find avg, and categorize those below ``low''
and above ``high'' - If no new values are needed for final
tables/graphs, please create 2 new variables anyway

\end{document}
